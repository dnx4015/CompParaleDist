%\input logo

%\vspace*{-3cm}

%\begin{figure}[h]
%\leavevmode
%\begin{minipage}[t]{\textwidth}
%\includegraphics[1cm,1cm][3cm,3cm]{logo-ufrpe.bmp}
%\end{minipage}
%\end{figure}



\vspace*{-2cm}
{\bf
\begin{center}
{\large
\hspace*{0cm}Universidade de S�o Paulo} \\
\hspace*{0cm}Instituto de Matem�tica e Estat�stica \\
\hspace*{0cm}Curso: Computa��o Paralela e Distribu�da  \\
\end{center}}
\vspace{4.0cm}
\noindent
\begin{center}
{\Large \bf Exerc�cio Programa 1: OpenMP} \\[3cm]
{\Large Autores:}\\[6mm]
{\Large Diana Naranjo}\\[6mm]
{\Large Walter Perez}\\[6mm]
\end{center}




{\raggedleft
\begin{minipage}[t]{8.3cm}
\setlength{\baselineskip}{0.25in}
Relat�rio do exerc�cio Programa 1, cujo objetivo e explorar a computac�o paralela com memoria compartilhada. Para isso foi usado o padr�o OpenMP. O desaf�o e usar a concis�o da sintaxe do OpenMP, mostrando que com poucos caracteres um programa ja pode ser paralelizado, ainda mais se a estrutura do codigo ajudar. 
\end{minipage}\\[2cm]}
\vspace{2cm}
{\center S�o Paulo \\[3mm]
Abril 2015 \\}


\newpage
